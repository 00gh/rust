% vim: tw=100

\documentclass[twocolumn]{article}
\usepackage{blindtext}
\usepackage[hypcap]{caption}
\usepackage{fontspec}
\usepackage[colorlinks, urlcolor={blue!80!black}]{hyperref}
\usepackage[outputdir=out]{minted}
\usepackage{relsize}
\usepackage{xcolor}

\setmonofont{Source Code Pro}[
  BoldFont={* Medium},
  BoldItalicFont={* Medium Italic},
  Scale=MatchLowercase,
]

\newcommand{\rust}[1]{\mintinline{rust}{#1}}

\begin{document}

\title{Miri: \\ \smaller{An interpreter for Rust's mid-level intermediate representation}}
\author{Scott Olson\footnote{\href{mailto:scott@solson.me}{scott@solson.me}} \\
  \smaller{Supervised by Christopher Dutchyn}}
\date{April 12th, 2016}
\maketitle

%%%%%%%%%%%%%%%%%%%%%%%%%%%%%%%%%%%%%%%%%%%%%%%%%%%%%%%%%%%%%%%%%%%%%%%%%%%%%%%%%%%%%%%%%%%%%%%%%%%%

\section{Abstract}

The increasing need for safe low-level code in contexts like operating systems and browsers is
driving the development of Rust\footnote{\url{https://www.rust-lang.org}}, a programming language
promising high performance without the risk of memory unsafety. To make programming more convenient,
it's often desirable to be able to generate code or perform some computation at compile-time. The
former is mostly covered by Rust's existing macro feature or build-time code generation, but the
latter is currently restricted to a limited form of constant evaluation capable of little beyond
simple math.

The architecture of the compiler at the time the existing constant evaluator was built limited its
potential for future extension. However, a new intermediate representation was recently
added\footnote{\href{https://github.com/rust-lang/rfcs/blob/master/text/1211-mir.md}{Rust RFC \#1211: Mid-level IR (MIR)}}
to the Rust compiler between the abstract syntax tree and the back-end LLVM IR, called mid-level
intermediate representation, or MIR for short. This report will demonstrate that writing an
interpreter for MIR is a surprisingly effective approach for supporting a large proportion of Rust's
features in compile-time execution.

%%%%%%%%%%%%%%%%%%%%%%%%%%%%%%%%%%%%%%%%%%%%%%%%%%%%%%%%%%%%%%%%%%%%%%%%%%%%%%%%%%%%%%%%%%%%%%%%%%%%

\section{Background}

The Rust compiler generates an instance of \rust{Mir} for each function [\autoref{fig:mir}]. Each
\rust{Mir} structure represents a control-flow graph for a given function, and contains a list of
``basic blocks'' which in turn contain a list of statements followed by a single terminator. Each
statement is of the form \rust{lvalue = rvalue}. An \rust{Lvalue} is used for referencing variables
and calculating addresses such as when dereferencing pointers, accessing fields, or indexing arrays.
An \rust{Rvalue} represents the core set of operations possible in MIR, including reading a value
from an lvalue, performing math operations, creating new pointers, structures, and arrays, and so
on. Finally, a terminator decides where control will flow next, optionally based on the value of a
boolean or integer.

\begin{figure}[ht]
  \begin{minted}[autogobble]{rust}
    struct Mir {
        basic_blocks: Vec<BasicBlockData>,
        // ...
    }

    struct BasicBlockData {
        statements: Vec<Statement>,
        terminator: Terminator,
        // ...
    }

    struct Statement {
        lvalue: Lvalue,
        rvalue: Rvalue
    }

    enum Terminator {
        Goto { target: BasicBlock },
        If {
            cond: Operand,
            targets: [BasicBlock; 2]
        },
        // ...
    }
  \end{minted}
  \caption{MIR (simplified)}
  \label{fig:mir}
\end{figure}

%%%%%%%%%%%%%%%%%%%%%%%%%%%%%%%%%%%%%%%%%%%%%%%%%%%%%%%%%%%%%%%%%%%%%%%%%%%%%%%%%%%%%%%%%%%%%%%%%%%%

\section{First implementation}

\subsection{Basic operation}

To investigate the possibility of executing Rust at compile-time I wrote an interpreter for MIR
called Miri\footnote{\url{https://github.com/tsion/miri}}. The structure of the interpreter closely
mirrors the structure of MIR itself. It starts executing a function by iterating the statement list
in the starting basic block, translating the lvalue into a pointer and using the rvalue to decide
what to write into that pointer. Evaluating the rvalue may involve reads (such as for the two sides
of a binary operation) or construction of new values. When the terminator is reached, it is used to
decide which basic block to jump to next. Finally, Miri repeats this entire process, reading
statements from the new block.

\subsection{Function calls}

To handle function call terminators\footnote{Calls occur only as terminators, never as rvalues.},
Miri is required to store some information in a virtual call stack so that it may pick up where it
left off when the callee returns. Each stack frame stores a reference to the \rust{Mir} for the
function being executed, its local variables, its return value location\footnote{Return value
pointers are passed in by callers.}, and the basic block where execution should resume. When Miri
encounters a \rust{Return} terminator in the MIR, it pops one frame off the stack and resumes the
previous function. Miri's execution ends when the function it was initially invoked with returns,
leaving the call stack empty.

It should be noted that Miri does not itself recurse when a function is called; it merely pushes a
virtual stack frame and jumps to the top of the interpreter loop. Consequently, Miri can interpret
deeply recursive programs without overflowing its native call stack. This approach would allow Miri
to set a virtual stack depth limit and report an error when a program exceeds it.

\subsection{Flaws}

This version of Miri supported quite a bit of the Rust language, including booleans, integers,
if-conditions, while-loops, structures, enums, arrays, tuples, pointers, and function calls,
requiring approximately 400 lines of Rust code. However, it had a particularly naive value
representation with a number of downsides. It resembled the data layout of a dynamic language like
Ruby or Python, where every value has the same size\footnote{An \rust{enum} is a discriminated union
with a tag and space to fit the largest variant, regardless of which variant it contains.} in the
interpreter:

\begin{minted}[autogobble]{rust}
  enum Value {
      Uninitialized,
      Bool(bool),
      Int(i64),
      Pointer(Pointer), // index into stack
      Aggregate {
        variant: usize,
        data: Pointer,
      },
  }
\end{minted}

This representation did not work well for aggregate types\footnote{That is, structures, enums,
arrays, tuples, and closures.} and required strange hacks to support them. Their contained values
were allocated elsewhere on the stack and pointed to by the aggregate value, which made it more
complicated to implement copying aggregate values from place to place.

Moreover, while the aggregate issues could be worked around, this value representation made common
unsafe programming tricks (which make assumptions about the low-level value layout) fundamentally
impossible.

%%%%%%%%%%%%%%%%%%%%%%%%%%%%%%%%%%%%%%%%%%%%%%%%%%%%%%%%%%%%%%%%%%%%%%%%%%%%%%%%%%%%%%%%%%%%%%%%%%%%

\section{Current implementation}

Roughly halfway through my time working on Miri, Eduard
Burtescu\footnote{\href{https://github.com/eddyb}{eddyb on GitHub}} from the Rust compiler
team\footnote{\url{https://www.rust-lang.org/team.html\#Compiler}} made a post on Rust's internal
forums about a ``Rust Abstract Machine''
specification\footnote{\href{https://internals.rust-lang.org/t/mir-constant-evaluation/3143/31}{Burtescu's
reply on ``MIR constant evaluation''}} which could be used to implement more powerful compile-time
function execution, similar to what is supported by C++14's \mintinline{cpp}{constexpr} feature.
After clarifying some of the details of the data layout with Burtescu via IRC, I started
implementing it in Miri.

\subsection{Raw value representation}

The main difference in the new value representation was to represent values by ``abstract
allocations'' containing arrays of raw bytes with different sizes depending on their types. This
mimics how Rust values are represented when compiled for physical machines. In addition to the raw
bytes, allocations carry information about pointers and undefined bytes.

\begin{minted}[autogobble]{rust}
  struct Memory {
      map: HashMap<AllocId, Allocation>,
      next_id: AllocId,
  }

  struct Allocation {
      bytes: Vec<u8>,
      relocations: BTreeMap<usize, AllocId>,
      undef_mask: UndefMask,
  }
\end{minted}

\subsubsection{Relocations}

The abstract machine represents pointers through ``relocations'', which are analogous to relocations
in linkers\footnote{\href{https://en.wikipedia.org/wiki/Relocation_(computing)}{Relocation
(computing) - Wikipedia}}. Instead of storing a global memory address in the raw byte representation
like on a physical machine, we store an offset from the start of the target allocation and add an
entry to the relocation table which maps the index of the offset bytes to the target allocation.

In \autoref{fig:reloc}, the relocation stored at offset 0 in \rust{y} points to offset 2 in \rust{x}
(the 2nd 16-bit integer). Thus, the relocation table for \rust{y} is \texttt{\{0 =>
x\}}, meaning the next $N$ bytes after offset 0 denote an offset into allocation \rust{x} where $N$
is the size of a pointer (4 in this example). The example shows this as a labelled line beneath the
offset bytes.

In effect, the abstract machine represents pointers as \rust{(allocation_id, offset)} pairs. This
makes it easy to detect when pointer accesses go out of bounds.

\begin{figure}[hb]
  \begin{minted}[autogobble]{rust}
    let x: [i16; 3] = [0xAABB, 0xCCDD, 0xEEFF];
    let y = &x[1];
    // x: BB AA DD CC FF EE (6 bytes)
    // y: 02 00 00 00 (4 bytes)
    //    └───(x)───┘
  \end{minted}
  \caption{Example relocation on 32-bit little-endian}
  \label{fig:reloc}
\end{figure}

\subsubsection{Undefined byte mask}

The final piece of an abstract allocation is the undefined byte mask. Logically, we store a boolean
for the definedness of every byte in the allocation, but there are multiple ways to make the storage
more compact. I tried two implementations: one based on the endpoints of alternating ranges of
defined and undefined bytes and the other based on a bitmask. The former is more compact but I found
it surprisingly difficult to update cleanly. I currently use the much simpler bitmask system.

See \autoref{fig:undef} for an example of an undefined byte in a value, represented by underscores.
Note that there is a value for the second byte in the byte array, but it doesn't matter what it is.
The bitmask would be $10_2$, i.e.\ \rust{[true, false]}.

\begin{figure}[hb]
  \begin{minted}[autogobble]{rust}
    let x: [u8; 2] = unsafe {
        [1, std::mem::uninitialized()]
    };
    // x: 01 __ (2 bytes)
  \end{minted}
  \caption{Example undefined byte}
  \label{fig:undef}
\end{figure}

\subsection{Computing data layout}

Currently, the Rust compiler's data layouts for types are hidden from Miri, so it does its own data
layout computation which will not always match what the compiler does, since Miri doesn't take
target type alignments into account. In the future, the Rust compiler may be modified so that Miri
can use the exact same data layout.

Miri's data layout calculation is a relatively simple transformation from Rust types to a structure
with constant size values for primitives and sets of fields with offsets for aggregate types. These
layouts are cached for performance.

%%%%%%%%%%%%%%%%%%%%%%%%%%%%%%%%%%%%%%%%%%%%%%%%%%%%%%%%%%%%%%%%%%%%%%%%%%%%%%%%%%%%%%%%%%%%%%%%%%%%

\section{Deterministic execution}
\label{sec:deterministic}

In order to be effective as a compile-time evaluator, Miri must have \emph{deterministic execution},
as explained by Burtescu in the ``Rust Abstract Machine'' post. That is, given a function and
arguments to that function, Miri should always produce identical results. This is important for
coherence in the type checker when constant evaluations are involved in types, such as for sizes of
array types:

\begin{minted}[autogobble,mathescape]{rust}
  const fn get_size() -> usize { /* $\ldots$ */ }
  let array: [i32; get_size()];
\end{minted}

Since Miri allows execution of unsafe code\footnote{In fact, the distinction between safe and unsafe
doesn't exist at the MIR level.}, it is specifically designed to remain safe while interpreting
potentially unsafe code. When Miri encounters an unrecoverable error, it reports it via the Rust
compiler's usual error reporting mechanism, pointing to the part of the original code where the
error occurred. Below is an example from Miri's
repository.\footnote{\href{https://github.com/tsion/miri/blob/master/test/errors.rs}{miri/test/errors.rs}}

\begin{minted}[autogobble]{rust}
  let b = Box::new(42);
  let p: *const i32 = &*b;
  drop(b);
  unsafe { *p }
  //       ~~ error: dangling pointer
  //            was dereferenced
\end{minted}
\label{dangling-pointer}

%%%%%%%%%%%%%%%%%%%%%%%%%%%%%%%%%%%%%%%%%%%%%%%%%%%%%%%%%%%%%%%%%%%%%%%%%%%%%%%%%%%%%%%%%%%%%%%%%%%%

\section{Language support}

In its current state, Miri supports a large proportion of the Rust language, detailed below. The
major exception is a lack of support for  FFI\footnote{Foreign Function Interface, e.g.\ calling
functions defined in Assembly, C, or C++.}, which eliminates possibilities like reading and writing
files, user input, graphics, and more. However, for compile-time evaluation in Rust, this limitation
is desired.

\subsection{Primitives}

Miri supports booleans, integers of various sizes and signed-ness (i.e.\ \rust{i8}, \rust{i16},
\rust{i32}, \rust{i64}, \rust{isize}, \rust{u8}, \rust{u16}, \rust{u32}, \rust{u64}, \rust{usize}),
and unary and binary operations over these types. The \rust{isize} and \rust{usize} types will be
sized according to the target machine's pointer size just like in compiled Rust. The \rust{char} and
float types (\rust{f32}, \rust{f64}) are not supported yet, but there are no known barriers to doing
so.

When examining a boolean in an \rust{if} condition, Miri will report an error if its byte
representation is not precisely 0 or 1, since having any other value for a boolean is undefined
behaviour in Rust. The \rust{char} type will have similar restrictions once it is implemented.

\subsection{Pointers}

Both references and raw pointers are supported, with essentially no difference between them in Miri.
It is also possible to do pointer comparisons and math. However, a few operations are considered
errors and a few require special support.

Firstly, pointers into the same allocations may be compared for ordering, but pointers into
different allocations are considered unordered and Miri will complain if you attempt this. The
reasoning is that different allocations may have different orderings in the global address space at
runtime, making this non-deterministic. However, pointers into different allocations \emph{may} be
compared for direct equality (they are always unequal).

Secondly, pointers represented using relocations may be compared against pointers casted from
integers (e.g.\ \rust{0 as *const i32}) for things like null pointer checks. To handle these cases,
Miri has a concept of ``integer pointers'' which are always unequal to abstract pointers. Integer
pointers can be compared and operated upon freely. However, note that it is impossible to go from an
integer pointer to an abstract pointer backed by a relocation. It is not valid to dereference an
integer pointer.

\subsubsection{Slice pointers}

Rust supports pointers to ``dynamically-sized types'' such as \rust{[T]} and \rust{str} which
represent arrays of indeterminate size. Pointers to such types contain an address \emph{and} the
length of the referenced array. Miri supports these fully.

\subsubsection{Trait objects}

Rust also supports pointers to ``trait objects'' which represent some type that implements a trait,
with the specific type unknown at compile-time. These are implemented using virtual dispatch with a
vtable, similar to virtual methods in C++. Miri does not currently support these at all.

\subsection{Aggregates}

Aggregates include types declared with \rust{struct} or \rust{enum} as well as tuples, arrays, and
closures. Miri supports all common usage of all of these types. The main missing piece is to handle
\texttt{\#[repr(..)]} annotations which adjust the layout of a \rust{struct} or \rust{enum}.

\subsection{Lvalue projections}

This category includes field accesses, dereferencing, accessing data in an \rust{enum} variant, and
indexing arrays. Miri supports all of these, including nested projections such as
\rust{*foo.bar[2]}.

\subsection{Control flow}

All of Rust's standard control flow features, including \rust{loop}, \rust{while}, \rust{for},
\rust{if}, \rust{if let}, \rust{while let}, \rust{match}, \rust{break}, \rust{continue}, and
\rust{return} are supported. In fact, supporting these was quite easy since the Rust compiler
reduces them all down to a small set of control-flow graph primitives in MIR.

\subsection{Function calls}

As previously described, Miri supports arbitrary function calls without growing the native stack
(only its virtual call stack). It is somewhat limited by the fact that cross-crate\footnote{A crate
is a single Rust library (or executable).} calls only work for functions whose MIR is stored in
crate metadata. This is currently true for \rust{const}, generic, and inline functions.
A branch of the compiler could be made that stores MIR for all functions. This would be a non-issue
for a compile-time evaluator based on Miri, since it would only call \rust{const fn}s.

\subsubsection{Method calls}

Miri supports trait method calls, including invoking all the compiler-internal lookup needed to find
the correct implementation of the method.

\subsubsection{Closures}

Calls to closures are also supported with the exception of one edge case\footnote{Calling a closure
that takes a reference to its captures via a closure interface that passes the captures by value is
not yet supported.}. The value part of a closure that holds the captured variables is handled as an
aggregate and the function call part is mostly the same as a trait method call, but with the added
complication that closures use a separate calling convention within the compiler.

\subsubsection{Function pointers}

Function pointers are not currently supported by Miri, but there is a relatively simple way they
could be encoded using a relocation with a special reserved allocation identifier. The offset of the
relocation would determine which function it points to in a special array of functions in the
interpreter.

\subsubsection{Intrinsics}

To support unsafe code, and in particular to support Rust's standard library, it became clear that
Miri would have to support calls to compiler
intrinsics\footnote{\url{https://doc.rust-lang.org/stable/std/intrinsics/index.html}}. Intrinsics
are function calls which cause the Rust compiler to produce special-purpose code instead of a
regular function call. Miri simply recognizes intrinsic calls by their unique
ABI\footnote{Application Binary Interface, which defines calling conventions. Includes ``C'',
``Rust'', and ``rust-intrinsic''.} and name and runs special-purpose code to handle them.

An example of an important intrinsic is \rust{size_of} which will cause Miri to write the size of
the type in question to the return value location. The Rust standard library uses intrinsics heavily
to implement various data structures, so this was a major step toward supporting them. Intrinsics
have been implemented on a case-by-case basis as tests which required them were written, and not all
intrinsics are supported yet.

\subsubsection{Generic function calls}

Miri needs special support for generic function calls since Rust is a \emph{monomorphizing}
compiler, meaning it generates a special version of each function for each distinct set of type
parameters it gets called with. Since functions in MIR are still polymorphic, Miri has to do the
same thing and substitute function type parameters into all types it encounters to get fully
concrete, monomorphized types. For example, in\ldots

\begin{minted}[autogobble]{rust}
  fn some<T>(t: T) -> Option<T> { Some(t) }
\end{minted}

\ldots{}Miri needs to know the size of \rust{T} to copy the right amount of bytes from the argument
to the return value. If we call \rust{some(10i32)} Miri will execute \rust{some} knowing that
\rust{T = i32} and generate a representation for \rust{Option<i32>}.

Miri currently does this monomorphization lazily on-demand unlike the Rust back-end which does it
all ahead of time.

\subsection{Heap allocations}

The next piece of the puzzle for supporting interesting programs (and the standard library) was heap
allocations. There are two main interfaces for heap allocation in Rust: the built-in \rust{Box}
rvalue in MIR and a set of C ABI foreign functions including \rust{__rust_allocate},
\rust{__rust_reallocate}, and \rust{__rust_deallocate}. These correspond approximately to
\mintinline{c}{malloc}, \mintinline{c}{realloc}, and \mintinline{c}{free} in C.

The \rust{Box} rvalue allocates enough space for a single value of a given type. This was easy to
support in Miri. It simply creates a new abstract allocation in the same manner as for
stack-allocated values, since there's no major difference between them in Miri.

The allocator functions, which are used to implement things like Rust's standard \rust{Vec<T>} type,
were a bit trickier. Rust declares them as \rust{extern "C" fn} so that different allocator
libraries can be linked in at the user's option. Since Miri doesn't actually support FFI and wants
full control of allocations for safety, it ``cheats'' and recognizes these allocator functions in
essentially the same way it recognizes compiler intrinsics. Then, a call to \rust{__rust_allocate}
simply creates another abstract allocation with the requested size and \rust{__rust_reallocate}
grows one.

In the future, Miri should also track which allocations came from \rust{__rust_allocate} so it can
reject reallocate or deallocate calls on stack allocations.

\subsection{Destructors}

When a value which ``owns'' some resource (like a heap allocation or file handle) goes out of scope,
Rust inserts \emph{drop glue} that calls the user-defined destructor for the type if it has one, and
then drops all of the subfields. Destructors for types like \rust{Box<T>} and \rust{Vec<T>}
deallocate heap memory.

Miri doesn't yet support calling user-defined destructors, but it has most of the machinery in place
to do so already. There \emph{is} support for dropping \rust{Box<T>} types, including deallocating
their associated allocations. This is enough to properly execute the dangling pointer example in
\autoref{sec:deterministic}.

\subsection{Constants}

Only basic integer, boolean, string, and byte-string literals are currently supported. Evaluating
more complicated constant expressions in their current form would be a somewhat pointless exercise
for Miri. Instead, we should lower constant expressions to MIR so Miri can run them directly, which
is precisely what would need be done to use Miri as the compiler's constant evaluator.

\subsection{Static variables}

Miri doesn't currently support statics, but they would need support similar to constants. Also note
that while it would be invalid to write to static (i.e.\ global) variables in Miri executions, it
would probably be fine to allow reads.

\subsection{Standard library}

Throughout the implementation of the above features, I often followed this process:

\begin{enumerate}
  \item Try using a feature from the standard library.
  \item See where Miri runs into stuff it can't handle.
  \item Fix the problem.
  \item Go to 1.
\end{enumerate}

At present, Miri supports a number of major non-trivial features from the standard library along
with tons of minor features. Smart pointer types such as \rust{Box}, \rust{Rc}\footnote{Reference
counted shared pointer} and \rust{Arc}\footnote{Atomically reference-counted thread-safe shared
pointer} all seem to work. I've also tested using the shared smart pointer types with \rust{Cell}
and \rust{RefCell}\footnote{\href{https://doc.rust-lang.org/stable/std/cell/index.html}{Rust
documentation for cell types}} for internal mutability, and that works as well, although
\rust{RefCell} can't ever be borrowed twice until I implement destructor calls, since a destructor
is what releases the borrow.

But the standard library collection I spent the most time on was \rust{Vec}, the standard
dynamically-growable array type, similar to C++'s \texttt{std::vector} or Java's
\texttt{java.util.ArrayList}. In Rust, \rust{Vec} is an extremely pervasive collection, so
supporting it is a big win for supporting a larger swath of Rust programs in Miri.

See \autoref{fig:vec} for an example (working in Miri today) of initializing a \rust{Vec} with a
small amount of space on the heap and then pushing enough elements to force it to reallocate its
data array. This involves cross-crate generic function calls, unsafe code using raw pointers, heap
allocation, handling of uninitialized memory, compiler intrinsics, and more.

\begin{figure}[t]
  \begin{minted}[autogobble]{rust}
    struct Vec<T> {
        data: *mut T,    // 4 byte pointer
        capacity: usize, // 4 byte integer
        length: usize,   // 4 byte integer
    }

    let mut v: Vec<u8> =
        Vec::with_capacity(2);
    // v: 00 00 00 00 02 00 00 00 00 00 00 00
    //    └─(data)──┘
    // data: __ __

    v.push(1);
    // v: 00 00 00 00 02 00 00 00 01 00 00 00
    //    └─(data)──┘
    // data: 01 __

    v.push(2);
    // v: 00 00 00 00 02 00 00 00 02 00 00 00
    //    └─(data)──┘
    // data: 01 02

    v.push(3);
    // v: 00 00 00 00 04 00 00 00 03 00 00 00
    //    └─(data)──┘
    // data: 01 02 03 __
  \end{minted}
  \caption{\rust{Vec} example on 32-bit little-endian}
  \label{fig:vec}
\end{figure}

Miri supports unsafe operations on \rust{Vec} like \rust{v.set_len(10)} or
\rust{v.get_unchecked(2)}, provided that such calls do no invoke undefined behaviour. If a call
\emph{does} invoke undefined behaviour, Miri will abort with an appropriate error message (see
\autoref{fig:vec-error}).

% You can even do unsafe things with \rust{Vec} like \rust{v.set_len(10)} or
% \rust{v.get_unchecked(2)}, but if you do these things carefully in a way that doesn't cause any
% undefined behaviour (just like when you write unsafe code for regular Rust), then Miri can handle it
% all. But if you do slip up, Miri will error out with an appropriate message (see
% \autoref{fig:vec-error}).

\begin{figure}[t]
  \begin{minted}[autogobble]{rust}
    fn out_of_bounds() -> u8 {
        let v = vec![1, 2];
        let p = unsafe { v.get_unchecked(5) };
        *p + 10
    //  ~~ error: pointer offset outside
    //       bounds of allocation
    }

    fn undefined_bytes() -> u8 {
        let v = Vec::<u8>::with_capacity(10);
        let p = unsafe { v.get_unchecked(5) };
        *p + 10
    //  ~~~~~~~ error: attempted to read
    //            undefined bytes
    }
  \end{minted}
  \caption{\rust{Vec} examples with undefined behaviour}
  \label{fig:vec-error}
\end{figure}

\newpage

Here is one final code sample Miri can execute that demonstrates many features at once, including
vectors, heap allocation, iterators, closures, raw pointers, and math:

\begin{minted}[autogobble]{rust}
  let x: u8 = vec![1, 2, 3, 4]
      .into_iter()
      .map(|x| x * x)
      .fold(0, |x, y| x + y);
    // x: 1e (that is, the hex value
    //        0x1e = 30 = 1 + 4 + 9 + 16)
\end{minted}

%%%%%%%%%%%%%%%%%%%%%%%%%%%%%%%%%%%%%%%%%%%%%%%%%%%%%%%%%%%%%%%%%%%%%%%%%%%%%%%%%%%%%%%%%%%%%%%%%%%%

\section{Future directions}

\subsection{Finishing the implementation}

There are a number of pressing items on my to-do list for Miri, including:

\begin{itemize}
  \item A much more comprehensive and automated test suite.
  \item User-defined destructor calls.
  \item Non-trivial casts between primitive types like integers and pointers.
  \item Handling statics and global memory.
  \item Reporting errors for all undefined behaviour.\footnote{\href{https://doc.rust-lang.org/reference.html\#behavior-considered-undefined}{The Rust reference on what is considered undefined behaviour}}
  \item Function pointers.
  \item Accounting for target machine primitive type alignment and endianness.
  \item Optimizations (undefined byte masks, tail-calls).
  \item Benchmarking Miri vs. unoptimized Rust.
  \item Various \texttt{TODO}s and \texttt{FIXME}s left in the code.
  \item Integrating into the compiler proper.
\end{itemize}

\subsection{Future projects}

Other possible Miri-related projects include:

\begin{itemize}
  \item A read-eval-print-loop (REPL) for Rust, which may be easier to implement on top of Miri than
    the usual LLVM back-end.
  \item A graphical or text-mode debugger that steps through MIR execution one statement at a time,
    for figuring out why some compile-time execution is raising an error or simply learning how Rust
    works at a low level.
  \item A less restricted version of Miri that is able to run foreign functions from C/C++ and
    generally has full access to the operating system. Such an interpreter could be used to more
    quickly prototype changes to the Rust language that would otherwise require changes to the LLVM
    back-end.
  \item Unit-testing the compiler by comparing the results of Miri's execution against the results
    of LLVM-compiled machine code's execution. This would help to guarantee that compile-time
    execution works the same as runtime execution.
  \item Some kind of Miri-based symbolic evaluator that examines multiple possible code paths at
    once to determine if undefined behaviour could be observed on any of them.
\end{itemize}

%%%%%%%%%%%%%%%%%%%%%%%%%%%%%%%%%%%%%%%%%%%%%%%%%%%%%%%%%%%%%%%%%%%%%%%%%%%%%%%%%%%%%%%%%%%%%%%%%%%%

\section{Final thoughts}

Writing an interpreter which models values of varying sizes, stack and heap allocation, unsafe
memory operations, and more requires some unconventional techniques compared to conventional
interpreters targeting dynamically-typed languages. However, aside from the somewhat complicated
abstract memory model, making Miri work was primarily a software engineering problem, and not a
particularly tricky one. This is a testament to MIR's suitability as an intermediate representation
for Rust---removing enough unnecessary abstraction to keep it simple. For example, Miri doesn't even
need to know that there are different kinds of loops, or how to match patterns in a \rust{match}
expression.

Another advantage to targeting MIR is that any new features at the syntax-level or type-level
generally require little to no change in Miri. For example, when the new ``question mark'' syntax
for error handling\footnote{
  \href{https://github.com/rust-lang/rfcs/blob/master/text/0243-trait-based-exception-handling.md}
    {Question mark syntax RFC}}
was added to rustc, Miri required no change to support it.
When specialization\footnote{
  \href{https://github.com/rust-lang/rfcs/blob/master/text/1210-impl-specialization.md}
    {Specialization RFC}}
was added, Miri supported it with just minor changes to trait method lookup.

Of course, Miri also has limitations. The inability to execute FFI and inline assembly reduces the
amount of Rust programs Miri could ever execute. The good news is that in the constant evaluator,
FFI can be stubbed out in cases where it makes sense, like I did with \rust{__rust_allocate}. For a
version of Miri not intended for constant evaluation, it may be possible to use libffi to call C
functions from the interpreter.

In conclusion, Miri is a surprisingly effective project, and a lot of fun to implement. Due to MIR's
tendency to collapse multiple source-level features into one, I often ended up supporting features I
hadn't explicitly intended to. I am excited to work with the compiler team going forward to try to
make Miri useful for constant evaluation in Rust.

%%%%%%%%%%%%%%%%%%%%%%%%%%%%%%%%%%%%%%%%%%%%%%%%%%%%%%%%%%%%%%%%%%%%%%%%%%%%%%%%%%%%%%%%%%%%%%%%%%%%

\section{Thanks}

A big thanks goes to Eduard Burtescu for writing the abstract machine specification and answering my
incessant questions on IRC, to Niko Matsakis for coming up with the idea for Miri and supporting my
desire to work with the Rust compiler, and to my research supervisor Christopher Dutchyn. Thanks
also to everyone else on the compiler team and on Mozilla IRC who helped me figure stuff out.

\end{document}
