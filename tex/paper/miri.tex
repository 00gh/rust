% vim: tw=100

\documentclass[twocolumn]{article}
\usepackage{blindtext}
\usepackage{fontspec}
\usepackage[colorlinks, urlcolor={blue!80!black}]{hyperref}
\usepackage{relsize}
\usepackage{xcolor}

\begin{document}

\title{Miri: \\ \smaller{An interpreter for Rust's mid-level intermediate representation}}
% \subtitle{test}
\author{Scott Olson\footnote{\href{mailto:scott@solson.me}{scott@solson.me}} \\
  \smaller{Supervised by Christopher Dutchyn}}
\date{April 8th, 2016}
\maketitle

\section{Abstract}

The increasing need for safe low-level code in contexts like operating systems and browsers is
driving the development of Rust\footnote{\url{https://www.rust-lang.org}}, a programming language
backed by Mozilla promising blazing speed without the segfaults. To make programming more
convenient, it's often desirable to be able to generate code or perform some computation at
compile-time. The former is mostly covered by Rust's existing macro feature, but the latter is
currently restricted to a limited form of constant evaluation capable of little beyond simple math.

When the existing constant evaluator was built, it would have been difficult to make it more
powerful than it is. However, a new intermediate representation was recently
added\footnote{\href{https://github.com/rust-lang/rfcs/blob/master/text/1211-mir.md}{The MIR RFC}}
to the Rust compiler between the abstract syntax tree and the back-end LLVM IR, called mid-level
intermediate representation, or MIR for short. As it turns out, writing an interpreter for MIR is a
surprisingly effective approach for supporting a large proportion of Rust's features in compile-time
execution.

\section{Motivation}

\blindtext

\section{First implementation}

% TODO(tsion): Find a place for this text.
Making Miri work was primarily an implementation problem. Writing an interpreter which models values
of varying sizes, stack and heap allocation, unsafe memory operations, and more requires some
unconventional techniques compared to many interpreters. Miri's execution remains safe even while
simulating execution of unsafe code, which allows it to detect when unsafe code does something
invalid.

\blindtext[2]

\section{Data layout}

\blindtext

\end{document}
